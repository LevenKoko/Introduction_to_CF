\documentclass[UTF8]{ctexart}
\usepackage[left=2.50cm, right=2.50cm, top=2.50cm, bottom=2.50cm]{geometry}
\usepackage{amsmath, amsfonts, amssymb}
\usepackage[english]{babel}
\usepackage{graphicx}
\usepackage{url}
\usepackage{bm}
\usepackage{indentfirst}
\usepackage{multirow}
% \usepackage{booktabs}
\usepackage{listings}
\usepackage{algorithm}
\usepackage{float}
\usepackage{algorithmic}
\usepackage{inconsolata}
\usepackage{color}
\usepackage{xcolor}
% \usepackage{abstract}
\usepackage{multicol}
\setlength{\parindent}{2em}
\CTEXsetup[format={\Large\bfseries}]{section}

\title{\textbf{一种基于智能合约的新型校园外卖体系探索}}
\author{\sffamily 张三,李四,王五\\ \sffamily(电子科技大学,四川 \ 成都)}
% \author{\sffamily 李四}
\date{}

\newcommand\picturehere[2][1]{\centerline{\includegraphics[scale = #1]{#2}}}
\newcommand\picfig[1]{\centerline{\small \heiti #1 \songti }}


\begin{document}
  \maketitle
  \textbf{\kaishu摘 \ 要:\heiti  \small 在疫情封锁下的大学校园内,校园外卖的取送效率已然成为难题,本文便以此为研究对象,通过对现有校园外卖平台存在的问题的发现,以智能合约技术以及相应的金融手段延时保险为基础,提出一种期望提高校园外卖效率的解决方案。目前校园外卖时间周期长的主要原因为中间流程分级多,校园内兼职骑手数量少且学业压力大。对此,我们希望提出一种点对点的人人可以充当骑手的取送系统,同时减少中间流程,以最大限度提升取送效率。同时“人人可送”的特点决定了该平台准入规则应对诚信要求度极高,因而我们选用了智能合约的方式来确保交易的顺利进行。另一方面,我们也通过研究现有大型外平台,建立了一套针对校园外卖的延时险模型与信用评估模型,借此进一步提高平台的效率问题。}
  \\
  \indent \textbf{\kaishu关键词:\heiti  \small 校园外卖,智能合约,延时保险,信用风险评估,效率}
  \pagestyle{plain}	% 去除页眉
\begin{multicols}{2}
  \setcounter{section}{-1}
  \section{引言}
  外卖平台和消费者都有对外卖延误险的需求,根据艾媒咨询发布的《2017-2018年中国在线餐饮外卖市场研究报告》\cite{None20182017}结果显示,“送餐速度”是影响用户选择外卖平台的第二大因素,第一名是食品安全保障,可见外卖消费者对配送的时效性要求很高。而截至2021年12月22日,我们研究了电子科技大学现有的三个外卖平台(空投闪送、科大闪送、成电校园易取送)均未设置外卖延时险的功能。\\
  \indent 事实上,这种情况出现的原因有三。一是因为校园外卖的平台管理者与配送的骑手本身为在校学生,学业任务繁重,加之校园特有的二次传输性质(见图1)。二是现有校园外卖延误频发,为此产生的红包赔偿是一笔巨大的开销。三是外卖平台与配送平台是两个不同的主体,在订单迟到后,外卖平台需要分别处理顾客的赔付、投诉工作和对配送平台的管理、培养骑手等合作洽谈工作,存在内部管理低效率的损失。而由于上述原因的存在,不仅电子科技大学,基本全国高校的校园外卖模式都存在以上痛点。
  \picturehere[0.6]{image/tradition.png}
  \picfig{图1 \ \ 传统校园外卖模式}%
  \\
  \indent 基于此,本文希望通过提出一种采用智能合约的新型校园外卖模式,让用户与骑手实现点对点的直接交易,减少中间过程,同时在合约内新增外卖延时险功能以激励骑手效率,最终达到大幅优化校园外卖传输速率的目的。
  \section{新型模式介绍}

  \subsection{骑手}
  \indent 在该设计平台中,与传统模式不同之处最大的便是骑手环节。\\
  \indent 考虑到校园周边商圈繁华,基本可以满足校园内学生的外卖需求,同时各个时段都有一定量的学生会在外活动并在某一时段返回学校。因此,这些同学的出入校行为均可以被认作是“闲置的资源”。若这些同学愿意接受在一定的报酬下进行帮助陌生同学将其下单所在店铺内的外卖订单带入校园并送至下单人寝室,则下单人(即用户顾客)可以获得更高效的外卖配送,“骑手”则可以在顺路回校的过程中获得更相应报酬。\\
  \picturehere[0.6]{image/new1.png}
  \picfig{图2  \ \ 简化后的流程示意图}
  \indent 当然,这里的“人人可送”是有先决条件的,而智能合约最重要的便是其安全性。如若让很多“不诚信”骑手接单,最后订单并没有安全按时的送到用户手里,最终会导致平台的可信度降低,口碑崩塌,用户减少。因而我们在后文将提出相应的准入制度、信用评级制度、延时险激励制度,以此来提高骑手的诚信度与积极性。\\
  \indent 理论上,一个骑手同时可以接许多订单,但其必须确保在其规定时间可以送到用户寝室,否则可能会被要求支付大量的延时赔付金额。\\
  \indent 为了确保骑手不会提前点击确认送达,每个骑手都会被分配一个属于他的独一无二的“骑手码”,只有骑手在将该二维码出示给用户且用户扫描后,才能算做该订单结束。
  \subsection{平台商家互动}
  \indent 在用户下单后,平台会提醒商家出单,同时会生成相应的订单码,骑手扫码后即可从商家领走该订单对应的商品。\\
  \indent 在该体系中,我们认为商家的出餐速度也会影响送餐的时间,故若外卖订单延时,赔偿金额应由平台、商家、骑手三方面共同承担。
  \subsection{平台用户互动}
  \indent 用户在下单过程中,可以查看当前想要购买的商家附近的可配送骑手人数,下单过后,系统自动根据骑手信息、商家信息以及当前配送条件生成预计配送时间\footnote{该部分超出本文讨论范围,暂不涉及。}。若用户购买延时险,则若订单延迟,平台会返还给用户与订单金额相关的一定赔偿。
  \picturehere[0.5]{image/contract.png}
  \picfig{图3 \ \ 合约与三方的互动}



  \section{信用评估和准入规则}
  \subsection{信用评估方式}
  \subsubsection{初始值}
  $$C_0=0.3\times C_p + 0.7 \times \frac{z}{950} \times 100$$

  $$C_0=\frac{z}{950} \times 100$$
  这里是初始值\\
  这里是初始值\\
  这里是初始值\\
  这里是初始值
  \subsubsection{动态评估}
  $$C_p^\prime=C_p-\ln(\frac{C_0}{65})\times e^2$$
  这里是动态评估
  \subsection{准入规则}
  \subsubsection{推荐制}
  \indent 本平台使用的推荐制是通过原本平台内的正式外卖员参与,经过他们的推荐与信用评估来完成新外卖员的纳新,但新外卖员的信用水平会一定程度上影响推荐者的信用水平。被推荐并通过评估的外卖员将成为正式外卖员,不需要提交押金即可进行接单。\\
  \indent “供求关系”是经济学的基石, 是指在商品经济条件下, 商品供给和需求之间的相互联系、相互制约的关系。外卖平台对优秀外卖员的争夺, 类似于商品经济中对商品的追求。“接收”平台相当于“买方”, 推荐用户相当于“卖方”, 新进外卖员则是推荐用户生产的“商品”。某一被推荐的外卖员, 热爱工作、积极进取, 则在市场上处于“供不应求”的地位;反之, 平台自身对外卖员待遇差, 优秀的外卖员自然不愿意与此平台签订合约, 必然导致“供大于求”。\\
  \indent 基于“供求关系”的分析可知, 推荐资格、推荐名额的分配, 以及接收推荐外卖员的比例, 应交由“市场需求”自主调节, 充分发挥“市场”调节的效率, 高效准确地选拔出优秀的外卖员。通过“市场”的优胜劣汰, 激发平台自身的发展潜力, 一方面, 推荐用户会高度重视自身的相关利益, 谨慎决定推荐数量与对象, 培养出符合平台需求的优秀外卖员。另一方面, 平台也会不断为新进外卖员提供优质的接单渠道等, 吸引更多的外卖员加入。
  \picturehere[0.45]{image/tree.png}
  \picfig{图4  \ \ 推荐制示意图}
  \subsubsection{押金制}
  \indent 本平台使用的押金制是等价抵押机制。非正式骑手可以通过提交与所接单价值等价的金额来完成接单,并在订单完成后与报酬一同汇入非正式骑手的账户中。随着此类订单完成次数的增加,非正式骑手的信用值会相应增加,随着信用值增加骑手所需缴纳的押金会逐渐减少,而信用值到一定程度后可申请成为正式骑手,此后接单将不再需要押金。\\
  \indent 押金能有效地保障消费者的权益,降低交易费用,从而保障交易安全。押金的存在避免了部分新进外卖员偷餐、毁坏订单等举动后销号退出的逃避后果的行为,是外卖员前期信用保证的辅助制度。
  \section{延时保险模型}
  \subsection{用户下单}
  \begin{table}[H]
  \centering
  \caption{变量解释}
  \begin{tabular}{|c|c|}
    \hline
  a           & 1           \\ \hline
  b           & 3           \\ \hline
  c           & 3           \\ \hline
  d           & 5           \\ \hline
  \end{tabular}
  \end{table}
%   \begin{table*}
%   \centering
%   \caption{Table of students.}
%   \begin{tabular}{|c|c|c|}
%   \hline
%   Name & Age & Gender  \\
%   Cat & 3 & Girl  \\
%   Dog & 5 & Boy \\
%   \hline
%   \end{tabular}
%   \label{table:stu}
% \end{table*}
  用户下单金额为$S$,期望骑手费为$R(R>R^{*})$,期望时间为$T$,用户需往平台中充入$S+R$。
  \subsection{商家接单}
  商家接单后需往平台中充入$R×\alpha(\alpha>1)$,出单时间为$t_1$(容忍时间为$t_1^{*}$)。
  \subsection{骑手接单}
     骑手接单后需往平台中充入$S×\beta(\beta>1)$,送达时间为$t_2$。
  \subsection{送达情况}
  \indent 送达后,则进行返还押金与分配利润的步骤。
  \subsubsection{成功送达}
    \indent下面讨论用户购买了延时保险的情况。 \\
    \indent 对商家有:
    $$
    M_s=\left\{
      \begin{array}{lr}
        R \times \alpha + S,  & {t_1 \leq t_1^{*}}\\
        R \times \alpha \times f(t_1-t_1^{*}) + S \times \gamma, & {t_1 > t_1^{*}}
      \end{array}\right.
    $$
    \indent 对骑手有:\\
    \indent令$$T_{md} = S \times \beta $$
    $$ G_{md}=R \times g(t_2-T+t_1^{*})$$
    \indent 则
    $$
    M_t=\left\{
      \begin{array}{lr}
        T_{md}+R,  & {t_2 \leq T-t_1^{*}}\\
        T_{md} + G_{md} + S \times \gamma, & {t_2 > T -t_1^{*}}
      \end{array}\right.
    $$
    \indent 对用户有:若$t_1+t_2>T$,则将合约内与该订单相关的钱悉数退回。
  \subsubsection{丢单}

  \indent 对商家有:
  $$M_s=R \times \alpha $$
  \indent 对骑手有:
  $$M_t=S \times \beta \times \sigma$$
  \indent 对用户有:
  $$M_c=S \times (1+\beta)+R \times (1+\alpha)-M_s-M_t$$
  \subsubsection{无人接单}
  \indent 用户可选择提高$R$以吸引其他骑手接单,或者选择取消订单,但需要支付一定的赔偿与手续费用于赔偿商家的损失。

\end{multicols}


\bibliographystyle{unsrt}
\bibliography{bib}
\end{document}
