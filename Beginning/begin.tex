\documentclass[UTF8]{ctexart}
\usepackage[left=2.50cm, right=2.50cm, top=2.50cm, bottom=2.50cm]{geometry} %ҳ�߾�
\usepackage{amsmath, amsfonts, amssymb} % ��ѧ��ʽ������
\usepackage[english]{babel}
\usepackage{graphicx}   % ͼƬ
\usepackage{url}        % ������
\usepackage{bm}         % �Ӵַ�������
\usepackage{indentfirst} %��������
\usepackage{multirow}
\usepackage{booktabs}
\usepackage{listings}
\usepackage{algorithm}
\usepackage{float}
\usepackage{algorithmic}
\usepackage{inconsolata}
\usepackage{color}
\usepackage{xcolor}
\usepackage{abstract}
\usepackage{multicol}
\setlength{\parindent}{2em}
\CTEXsetup[format={\Large\bfseries}]{section}
\title{\textbf{一种基于智能合约的新型校园外卖体系探索}}
\author{\sffamily 张三,李四,王五\\ \sffamily(电子科技大学,四川 \ 成都)}
% \author{\sffamily 李四}
\date{}

\newcommand\picturehere[2][1]{\centerline{\includegraphics[scale = #1]{#2}}}
\newcommand\picfig[1]{\begin{center}\small \heiti #1 \songti \end{center}}


\begin{document}
  \maketitle
  \textbf{\kaishu摘 \ 要:\heiti  \small 在疫情封锁下的大学校园内,校园外卖的取送效率已然成为难题,本文便以此为研究对象,通过对现有校园外卖平台存在的问题的发现,以智能合约技术以及相应的金融手段延时保险为基础,提出一种期望提高校园外卖效率的解决方案。目前校园外卖时间周期长的主要原因为中间流程分级多,校园内兼职骑手数量少且学业压力大。对此,我们希望提出一种点对点的人人可以充当骑手的取送系统,同时减少中间流程,以最大限度提升取送效率。同时“人人可送”的特点决定了该平台准入规则应对诚信要求度极高,因而我们选用了智能合约的方式来确保交易的顺利进行。另一方面,我们也通过研究现有大型外平台,建立了一套针对校园外卖的延时险模型与信用评估模型,借此进一步提高平台的效率问题。}
  \\
  \indent
  \textbf{\kaishu关键词:\heiti  \small 校园外卖,智能合约,延时保险,信用风险评估,效率}
\begin{multicols}{2}
  \setcounter{section}{-1}
  \section{引言}
  外卖平台和消费者都有对外卖延误险的需求,根据艾媒咨询发布的《2017-2018年中国在线餐饮外卖市场研究报告》\cite{None20182017}结果显示,“送餐速度”是影响用户选择外卖平台的第二大因素,第一名是食品安全保障,可见外卖消费者对配送的时效性要求很高。而截至2021年12月22日,我们研究了电子科技大学现有的三个外卖平台(空投闪送、科大闪送、成电校园易取送)均未设置外卖延时险的功能。\\
  \indent 事实上,这种情况出现的原因有三。一是因为校园外卖的平台管理者与配送的骑手本身为在校学生,学业任务繁重,加之校园特有的二次传输性质(见图1)。二是现有校园外卖延误频发,为此产生的红包赔偿是一笔巨大的开销。三是外卖平台与配送平台是两个不同的主体,在订单迟到后,外卖平台需要分别处理顾客的赔付、投诉工作和对配送平台的管理、培养骑手等合作洽谈工作,存在内部管理低效率的损失。而由于上述原因的存在,不仅电子科技大学,基本全国高校的校园外卖模式都存在以上痛点。
  \picturehere[0.6]{image/tradition.png}
  \picfig{图1 \ \ 传统校园外卖模式}%
  \indent 基于此,本文希望通过提出一种采用智能合约的新型校园外卖模式,让用户与骑手实现点对点的直接交易,减少中间过程,同时在合约内新增外卖延时险功能以激励骑手效率,最终达到大幅优化校园外卖传输速率的目的。
  \section{新型模式介绍}
  这里是新型模式介绍
  \section{信用评估和准入规则}
  这里是信用评估和准入规则
\end{multicols}


\bibliographystyle{unsrt}
\bibliography{bib}
\end{document}
