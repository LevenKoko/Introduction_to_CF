\documentclass[UTF8]{ctexart}
\usepackage[left=2.50cm, right=2.50cm, top=2.50cm, bottom=2.50cm]{geometry}
\usepackage{amsmath, amsfonts, amssymb}
\usepackage[english]{babel}
\usepackage{graphicx}
\usepackage{url}
\usepackage{bm}
\usepackage{indentfirst}
\usepackage{threeparttable}
\usepackage{multirow}
\usepackage{cite}
% \usepackage{booktabs}
\usepackage{listings}
\usepackage{algorithm}
\usepackage{float}
\usepackage{algorithmic}
\usepackage{inconsolata}
\usepackage{color}
\usepackage{xcolor}
% \usepackage{abstract}
\usepackage{multicol}
\setlength{\parindent}{2em}
\CTEXsetup[format={\Large\bfseries}]{section}

\title{\textbf{一种基于智能合约的新型校园外卖体系探索}}
\author{\sffamily 张三,李四,王五\\ \sffamily(电子科技大学,四川 \ 成都)}
% \author{\sffamily 李四}
\date{}

\newcommand\picturehere[2][1]{\centerline{\includegraphics[scale = #1]{#2}}}
\newcommand\picfig[1]{\centerline{\small \heiti #1 \songti }}

\renewcommand\tablename{表}

\begin{document}
  \maketitle
  \textbf{\kaishu摘 \ 要:\heiti  \small 在疫情封锁下的大学校园内,校园外卖的取送效率已然成为难题,本文便以此为研究对象,通过对现有校园外卖平台存在的问题的发现,以智能合约技术以及相应的金融手段延时保险为基础,提出一种期望提高校园外卖效率的解决方案。目前校园外卖时间周期长的主要原因为中间流程分级多,校园内兼职骑手数量少且学业压力大。对此,我们希望提出一种点对点的人人可以充当骑手的取送系统,同时减少中间流程,以最大限度提升取送效率。同时“人人可送”的特点决定了该平台准入规则应对诚信要求度极高,因而我们选用了智能合约的方式来确保交易的顺利进行。另一方面,我们也通过研究现有大型外平台,建立了一套针对校园外卖的延时险模型与信用评估模型,借此进一步提高平台的效率问题。}
  \\
  \indent \textbf{\kaishu关键词:\heiti  \small 校园外卖,智能合约,延时保险,信用风险评估,效率}
  \pagestyle{plain}	% 去除页眉
\begin{multicols}{2}
  \setcounter{section}{-1}
  \section{引言}
  外卖平台和消费者都有对外卖延误险的需求,根据艾媒咨询发布的《2017-2018年中国在线餐饮外卖市场研究报告》\cite{None20182017}结果显示,“送餐速度”是影响用户选择外卖平台的第二大因素,第一名是食品安全保障,可见外卖消费者对配送的时效性要求很高。而截至2021年12月22日,我们研究了电子科技大学现有的三个外卖平台(空投闪送、科大闪送、成电校园易取送)均未设置外卖延时险的功能。\\
  \indent 事实上,这种情况出现的原因有三。一是因为校园外卖的平台管理者与配送的骑手本身为在校学生,学业任务繁重,加之校园特有的二次传输性质(见图1)。二是现有校园外卖延误频发,为此产生的红包赔偿是一笔巨大的开销。三是外卖平台与配送平台是两个不同的主体,在订单迟到后,外卖平台需要分别处理顾客的赔付、投诉工作和对配送平台的管理、培养骑手等合作洽谈工作,存在内部管理低效率的损失。而由于上述原因的存在,不仅电子科技大学,基本全国高校的校园外卖模式都存在以上痛点。
  \picturehere[0.6]{image/tradition.png}
  \picfig{图1 \ \ 传统校园外卖模式}%
  \\
  \indent 基于此,本文希望通过提出一种采用智能合约的新型校园外卖模式,让用户与骑手实现点对点的直接交易,减少中间过程,同时在合约内新增外卖延时险功能以激励骑手效率,最终达到大幅优化校园外卖传输速率的目的。
  \section{新型模式介绍}

  \subsection{骑手}
  \indent 在该设计平台中,与传统模式不同之处最大的便是骑手环节。\\
  \indent 考虑到校园周边商圈繁华,基本可以满足校园内学生的外卖需求,同时各个时段都有一定量的学生会在外活动并在某一时段返回学校。因此,这些同学的出入校行为均可以被认作是“闲置的资源”。若这些同学愿意接受在一定的报酬下进行帮助陌生同学将其下单所在店铺内的外卖订单带入校园并送至下单人寝室,则下单人(即用户顾客)可以获得更高效的外卖配送,“骑手”则可以在顺路回校的过程中获得更相应报酬。\\
  \picturehere[0.6]{image/new1.png}
  \picfig{图2  \ \ 简化后的流程示意图}
  \indent 当然,这里的“人人可送”是有先决条件的,而智能合约最重要的便是其安全性。如若让很多“不诚信”骑手接单,最后订单并没有安全按时的送到用户手里,最终会导致平台的可信度降低,口碑崩塌,用户减少。因而我们在后文将提出相应的准入制度、信用评级制度、延时险激励制度,以此来提高骑手的诚信度与积极性。\\
  \indent 理论上,一个骑手同时可以接许多订单,但其必须确保在其规定时间可以送到用户寝室,否则可能会被要求支付大量的延时赔付金额。\\
  \indent 为了确保骑手不会提前点击确认送达,每个骑手都会被分配一个属于他的独一无二的“骑手码”,只有骑手在将该二维码出示给用户且用户扫描后,才能算做该订单结束。
  \subsection{平台商家互动}
  \indent 在用户下单后,平台会提醒商家出单,同时会生成相应的订单码,骑手扫码后即可从商家领走该订单对应的商品。\\
  \indent 在该体系中,我们认为商家的出餐速度也会影响送餐的时间,故若外卖订单延时,赔偿金额应由平台、商家、骑手三方面共同承担。
  \subsection{平台用户互动}
  \indent 用户在下单过程中,可以查看当前想要购买的商家附近的可配送骑手人数,下单过后,系统自动根据骑手信息、商家信息以及当前配送条件生成预计配送时间\footnote{该部分超出本文讨论范围,暂不涉及。}。若用户购买延时险,则若订单延迟,平台会返还给用户与订单金额相关的一定赔偿。
  \picturehere[0.5]{image/contract.png}
  \picfig{图3 \ \ 合约与三方的互动}



  \section{信用评估和准入规则}
  \begin{table}[H]
    \centering
    \begin{tabular}{|c|c|}
      \hline
      $C_0$ & 被推荐用户初始信用 \\ \hline
      $C_p$ & 推荐者的信用 \\ \hline
      z &被推荐者的芝麻分\footnotemark[2]\\ \hline
      $C_p^{'}$ & 被推荐者初始信用 \\ \hline
      x & 用户评分 \\ \hline
      N & 同时接受订单数量 \\ \hline
    \end{tabular}
  \end{table}
  \footnotetext[2]{支付宝提供了芝麻分查询的接口,这不但可以实现信用的评估,还可以实现一人一个账号的限制功能。}
  \picfig{表格1 \ \ 信用评估与准入}
  \subsection{信用评估方式}
  \subsubsection{初始值}
  \indent 推荐制是由原用户推荐,新用户得以加入的机制。推荐制进入的用户虽仍要在接单时缴纳押金,但是将享有特别的福利,此环节将在“动态评估”部分讨论。
  \indent 通过推荐制成为外卖员的用户,初始的信用分将由推荐者的信用、被推荐者的支付宝芝麻分决定:
  $$C_0=0.3 C_p + 0.7 \times \frac{z}{950} \times 100$$
  \indent 押金制原指客户在买卖期货时需缴纳相当于合同价值一定比例的押金,在这里指的是通过缴纳一定的押金作为保证金,作为用户初期信用与消费者权益的保证。\\
  \indent通过押金制成为外卖员的用户,初始的信用分数将由其支付宝芝麻分决定,公式如下:
  $$C_0=\frac{z}{950} \times 100$$
  \subsubsection{动态评估}
  对于推荐制,推荐者初始每人有3个推荐名额,且推荐者的信用分将会受到被推荐者的影响:
  $$C_p^\prime=C_p-\ln(\frac{C_0}{65})\times e^2$$
  \indent 随着信用值的上升,推荐者将获得更多的推荐名额n:
  $$n=\lfloor 3+\frac{C_p-65}{187} \rfloor ,(n \leq 10)$$
  \indent 而被推荐者也将享受到推荐制的福利,被推荐后将享有5单的免押金订单的机会,且将获得信用积累加成:
  $$\mathrm{\Delta C}^\prime=(1+\frac{10}{x_t} )\mathrm{\Delta C}$$
  \indent 而对于一般用户,每当用户完成订单且获得好评(4星/5星)时,其信用分C就能提升$\Delta C$
  $$\Delta C=\left\{
    \begin{array}{lr}
    \frac{x-3}{5} N, & {N \leq 5} \\
    \frac{x-3}{5} N + \frac{x-3}{5}\times \frac{3N-15}{N} ,& {N > 5}
  \end{array}\right.
  $$
  \indent 如果获得中评,则信用积分不发生改变;\\
  \indent 但如果获得了差评(1星/2星)时,其信用分C就会下降$\Delta C$
  $$\Delta C=\left\{
    \begin{array}{lr}
    \frac{x-3}{5} N, & {N \leq 5} \\
    \frac{x-3}{5} N + \frac{x-3}{5}\times \frac{3N-15}{N}, & {N > 5}
  \end{array}\right.
$$
  \indent 其中x为用户评分,N为同时接受的订单数。此处限制了骑手同时接多单所获得的信用值,是为了避免骑手所接单数超出其能力范围而导致延误的情况再度出现。\\
  \indent 但当订单出现较大的事故(收到顾客投诉等情况)时,经核实后将会扣除△C的信用,并且此后所接五单价值无法超过30元。
  $$\mathrm{\Delta C}=-\left(100+\frac{c_0}{10}\right)$$
  \subsubsection{信用等级制度}
  \indent 信用是衡量一位外卖员诚信度与活跃度的重要指标。用户成功完成的订单数越多,信用将会越高,反之当出现丢单或延误等顾客投诉的情况,用户信用将较大幅度降低。\\
  \indent 信用的最大值为1000。每100分为1等级,且用户凭借等级lv能够获得押金减免D:
  $$lv=\lfloor \frac{C_0}{100}\rfloor$$
  $$D=10lv$$
 同时,信用等级高的用户将获得额外的分红:
$$\Delta m=\frac{lv}{20}\times\ \Delta m_p$$
其中$\Delta m_p$是平台在此单中获得的收益。

  \subsection{准入规则}
  \subsubsection{推荐制}
  \indent 本平台使用的推荐制是通过原本平台内的正式外卖员参与,经过他们的推荐与信用评估来完成新外卖员的纳新,但新外卖员的信用水平会一定程度上影响推荐者的信用水平。被推荐并通过评估的外卖员将成为正式外卖员,不需要提交押金即可进行接单。\\
  \indent “供求关系”是经济学的基石, 是指在商品经济条件下, 商品供给和需求之间的相互联系、相互制约的关系。外卖平台对优秀外卖员的争夺, 类似于商品经济中对商品的追求。“接收”平台相当于“买方”, 推荐用户相当于“卖方”, 新进外卖员则是推荐用户生产的“商品”。某一被推荐的外卖员, 热爱工作、积极进取, 则在市场上处于“供不应求”的地位;反之, 平台自身对外卖员待遇差, 优秀的外卖员自然不愿意与此平台签订合约, 必然导致“供大于求”。\\
  \indent 基于“供求关系”的分析可知, 推荐资格、推荐名额的分配, 以及接收推荐外卖员的比例, 应交由“市场需求”自主调节, 充分发挥“市场”调节的效率, 高效准确地选拔出优秀的外卖员。通过“市场”的优胜劣汰, 激发平台自身的发展潜力, 一方面, 推荐用户会高度重视自身的相关利益, 谨慎决定推荐数量与对象, 培养出符合平台需求的优秀外卖员。另一方面, 平台也会不断为新进外卖员提供优质的接单渠道等, 吸引更多的外卖员加入。
  \picturehere[0.4]{image/tree.png}
  \picfig{图4  \ \ 推荐制示意图}
  \subsubsection{押金制}
  \indent 本平台使用的押金制是等价抵押机制。非正式骑手可以通过提交与所接单价值等价的金额来完成接单,并在订单完成后与报酬一同汇入非正式骑手的账户中。随着此类订单完成次数的增加,非正式骑手的信用值会相应增加,随着信用值增加骑手所需缴纳的押金会逐渐减少,而信用值到一定程度后可申请成为正式骑手,此后接单将不再需要押金。\\
  \indent 押金能有效地保障消费者的权益,降低交易费用,从而保障交易安全。押金的存在避免了部分新进外卖员偷餐、毁坏订单等举动后销号退出的逃避后果的行为,是外卖员前期信用保证的辅助制度。
  \subsubsection{实际结合方案}
  经过讨论,我们发现,两种方法单独使用均无法达到使平台健康发展的目的,仅使用推荐制会导致骑手数量无法扩张且无法实现“人人都有机会参与”的初衷;而仅使用押金制则无法树立平台的诚信,常常被误认为是诈骗,而导致平台成立初期无法发展。\\
  \indent 因此本平台最终使用的准入规则是推荐制与押金制并行、分级准入的准则。\\
  \indent 在平台发展的初期,无法获得足够的人力,而押金制在信用建立的前期很难推行。故在平台开设初期应先招募一批愿意参与试运行的外卖员,先由他们通过试单来体验平台的收益,并通过推荐制来招募新的人员,此时推荐制的信用激励机制则起到了助推的作用。在平台的信用建立起来之后,再逐步形成以押金制为主,以推荐制为辅的运行方式。通过学号+电话号码的验证机制与初始信用分的评估机制,每一位学生都有机会通过上述两种方式参与到代送外卖的交易中,并随着参与次数与推荐成员的增加获得更多的福利,进而使更多学生参与到本平台中。由此,外卖由于人手不足导致的送餐延误将会大大减少。


  \section{延时保险模型}
  \begin{table}[H]
    \centering
    \begin{tabular}{|c |c|}
      \hline
      R               & 期望骑手费用           \\ \hline
      S               & 订单金额           \\ \hline
      $R^{*}$         & 系统计算骑手费用最小值           \\ \hline
      $t_1$           & 商家出单时间           \\ \hline
      $t_1^{*}$       & 系统计算商家出单时间           \\ \hline
      $t_2$           & 骑手送餐时间       \\ \hline
      $M_s$           & 返回商家金额           \\ \hline
      $M_t$           & 返还骑手金额           \\ \hline
      $M_c$           & 返还顾客金额           \\ \hline
      $\alpha,\beta,\gamma,\sigma$           & 赔付系数(在正文中解释)           \\ \hline
    \end{tabular}
    % \caption{}
  \end{table}
  \picfig{表格2 \  延时保险赔付模型}
  \subsection{用户下单}
  用户下单金额为$S$,期望骑手费为$R(R>R^{*})$,期望时间为$T$,用户需往平台中充入$S+R$。
  \subsection{商家接单}
  商家接单后需往平台中充入$R×\alpha(\alpha>1)$,出单时间为$t_1$(容忍时间为$t_1^{*}$)。
  \subsection{骑手接单}
     骑手接单后需往平台中充入$S×\beta(\beta>1)$,送达时间为$t_2$。
  \subsection{送达情况}
  \indent 送达后,则进行返还押金与分配利润的步骤。
  \subsubsection{成功送达}
    \indent下面讨论用户购买了延时保险的情况。 \\
    \indent 对商家有:
    $$
    M_s=\left\{
      \begin{array}{lr}
        R \times \alpha + S,  & {t_1 \leq t_1^{*}}\\
        R \times \alpha \times f(t_1-t_1^{*}) + S \times \gamma, & {t_1 > t_1^{*}}
      \end{array}\right.
    $$
    \indent 对骑手有:\\
    \indent令$$T_{md} = S \times \beta $$
    $$ G_{md}=R \times g(t_2-T+t_1^{*})$$
    \indent 则
    $$
    M_t=\left\{
      \begin{array}{lr}
        T_{md}+R,  & {t_2 \leq T-t_1^{*}}\\
        T_{md} + G_{md} + S \times \gamma, & {t_2 > T -t_1^{*}}
      \end{array}\right.
    $$
    \indent 对用户有:若$t_1+t_2>T$,则将合约内与该订单相关的钱悉数退回。
  \subsubsection{丢单}

  \indent 对商家有:
  $$M_s=R \times \alpha $$
  \indent 对骑手有:
  $$M_t=S \times \beta \times \sigma$$
  \indent 对用户有:
  $$M_c=(1+\alpha)\times R +(1+\beta)\times S    -M_s-M_t$$
  \subsubsection{无人接单}
  \indent 用户可选择提高$R$以吸引其他骑手接单,或者选择取消订单,但需要支付一定的赔偿与手续费用于赔偿商家的损失。
  \subsection{赔付系数}
  \indent 这里我们令$h(n)$为单位次数商家发生延迟的比率,$p(m)$为单位次数骑手延迟的比率。\\
  \indent 再令:$$E(n)=1-e^{- \int_{0}^{n}h(n)dn}$$
  $$F(m)=1-e^{- \int_{0}^{m}p(m)dm}$$
  \indent 则以上系数$\alpha, \beta$ 分别可以表示为:
  $$\alpha = \frac{E(n)+E_A}{E_A}$$
  $$\beta = \frac{F(m)+F_A}{F_A}$$
  \indent 另外的$\gamma$则决定的是成功送达时平台的盈利系数,其与骑手用户的信用等级、订单金额等因素均相关,且在平台不同阶段均会产生改变,暂设定为人为改变系数,这里不做讨论。\\
  \indent 而$\sigma$则是丢单时骑手应赔付的金额,这与通知用户丢单的时间造成的时间损失与用户订单的金额相关,同时视外界环境因素(如恶劣天气)影响,这里也暂不做讨论。

\section{完备性与缺点分析}
\subsection{完备性分析}
\indent 下面将经由与传统校园外卖模式的对比阐明本模型的完备性。\\
\indent 其一,现有校园外卖系统普遍缺少延时险功能,难以基于外卖延误等问题对消费者进行赔偿。该模型运用智能合约平台提供了完善的延时险机制,可以有效地对消费者进行补偿。同时,商家和骑手双方面临的补偿金额,也能在一定程度上引起他们对于配送效率和配送安全的重视,减少校园内送餐拖延事件的发生。\\
\indent其二,我们看到,即使一些全国大型外卖平台提供了相应的延时险服务,但是目前的外卖延时险制度细则和标准各不相同,具体实践存在差异。同时由于在传统的延时险赔偿中,监管者往往处于被动监管地位,消费者处于弱势地位,外卖企业对于延时纠纷的判定从主观上更加倾向于自身利益。基于智能合约的延时险赔偿机制则可以做到判定的公平,从客观上消除企业对消费者这种隐性的利益损害。\\
\indent其三,目前的外卖信息管理都是中心化的,外卖企业对于外卖配送数据处于支配地位,客观上存在数据篡改、数据伪造等可能,从而影响纠纷判定。这种情况将在很大程度上导致消费者对外卖平台、商家的不信任以及对现有外卖延时险机制的不满,同时也增加了监管者对纠纷的监管难度。我们通过搭建去中心化的平台,客观上去除了数据篡改与伪造的可能性,智能合约代码公开的特性也一定程度上解决了平台与消费者、商家与消费者之间的信任问题。\\
\indent其四,区块链不可篡改的特性极其适用于配送流程追溯领域,其引入使得外卖过程中的数据更具备真实性。并且在区块链上,触发延时险赔偿的原因细节、赔付资金流向等信息能够实现公开透明,打破外卖企业与消费者之间的信息不对等,进一步维护消费者权益。\cite{fuck}
\subsection{缺陷分析}
\indent到目前为止,我们的模型优化了现有的校园外卖系统,但是对于该平台本身来说,它在外卖延时险服务中往往只能接受来自消费者的反馈,无法主动参与监管,处于间接监督的位置。换句话说,只有当延时行为发生,平台才能做出相应的反应。为了更加主动更有效率地监督外卖配送业务,信息化的管理便必不可少,新的监管方式和技术手段也必须运用到外卖配送领域,将外卖配送中的关键环节记录下来,在保证商业隐私地情况下做到公开透明,以便减少“外卖事故”的发生。\\
\indent另外,由于智能合约一经部署便无法进行改动,所以平台本身将面临调整与修复上的困难。其中许多参数需要随着时间与经营环境的改变进行调整,以便适应市场变化;同时,考虑到平台可能存在难以发现的漏洞,在未来时时面临着被漏洞攻击的可能性,我们也需要根据情况对智能合约进行修改与完善。无论是调整还是修复,都需要将更新后的合约重新部署到区块链上,而平台与合约之间的接口也需要随着合约地址的改变而改变,这些都将促成维护成本的上升。
\subsection{与传统模式的对比}
\picturehere[0.55]{image/advance.png}
\picfig{图5  \ \  改进模型对比}


\end{multicols}

\renewcommand{\refname}{参考文献}
\bibliographystyle{plain}
\bibliography{bib}
\end{document}
